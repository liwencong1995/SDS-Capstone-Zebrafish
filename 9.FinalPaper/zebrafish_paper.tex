% Template for PLoS
% Version 3.4 January 2017
\documentclass[10pt,letterpaper]{article}
\usepackage[top=0.85in,left=2.75in,footskip=0.75in]{geometry}

% amsmath and amssymb packages, useful for mathematical formulas and symbols
\usepackage{amsmath,amssymb}

% Use adjustwidth environment to exceed column width (see example table in text)
\usepackage{changepage}

% Use Unicode characters when possible
\usepackage[utf8x]{inputenc}

% textcomp package and marvosym package for additional characters
\usepackage{textcomp,marvosym}

% cite package, to clean up citations in the main text. Do not remove.
% \usepackage{cite}

% Use nameref to cite supporting information files (see Supporting Information section for more info)
\usepackage{nameref,hyperref}

% line numbers
\usepackage[right]{lineno}

% ligatures disabled
\usepackage{microtype}
\DisableLigatures[f]{encoding = *, family = * }

% color can be used to apply background shading to table cells only
\usepackage[table]{xcolor}

% array package and thick rules for tables
\usepackage{array}

% create "+" rule type for thick vertical lines
\newcolumntype{+}{!{\vrule width 2pt}}

% create \thickcline for thick horizontal lines of variable length
\newlength\savedwidth
\newcommand\thickcline[1]{%
  \noalign{\global\savedwidth\arrayrulewidth\global\arrayrulewidth 2pt}%
  \cline{#1}%
  \noalign{\vskip\arrayrulewidth}%
  \noalign{\global\arrayrulewidth\savedwidth}%
}

% \thickhline command for thick horizontal lines that span the table
\newcommand\thickhline{\noalign{\global\savedwidth\arrayrulewidth\global\arrayrulewidth 2pt}%
\hline
\noalign{\global\arrayrulewidth\savedwidth}}


% Remove comment for double spacing
%\usepackage{setspace} 
%\doublespacing

% Text layout
\raggedright
\setlength{\parindent}{0.5cm}
\textwidth 5.25in 
\textheight 8.75in

% Bold the 'Figure #' in the caption and separate it from the title/caption with a period
% Captions will be left justified
\usepackage[aboveskip=1pt,labelfont=bf,labelsep=period,justification=raggedright,singlelinecheck=off]{caption}
\renewcommand{\figurename}{Fig}

% Use the PLoS provided BiBTeX style
% \bibliographystyle{plos2015}

% Remove brackets from numbering in List of References
\makeatletter
\renewcommand{\@biblabel}[1]{\quad#1.}
\makeatother

% Leave date blank
\date{}

% Header and Footer with logo
\usepackage{lastpage,fancyhdr,graphicx}
\usepackage{epstopdf}
\pagestyle{myheadings}
\pagestyle{fancy}
\fancyhf{}
\setlength{\headheight}{27.023pt}
\lhead{\includegraphics[width=2.0in]{PLOS-submission.eps}}
\rfoot{\thepage/\pageref{LastPage}}
\renewcommand{\footrule}{\hrule height 2pt \vspace{2mm}}
\fancyheadoffset[L]{2.25in}
\fancyfootoffset[L]{2.25in}
\lfoot{\sf PLOS}

%% Include all macros below
\newcommand{\lorem}{{\bf LOREM}}
\newcommand{\ipsum}{{\bf IPSUM}}

\usepackage{color}
\usepackage{fancyvrb}
\newcommand{\VerbBar}{|}
\newcommand{\VERB}{\Verb[commandchars=\\\{\}]}
\DefineVerbatimEnvironment{Highlighting}{Verbatim}{commandchars=\\\{\}}
% Add ',fontsize=\small' for more characters per line
\usepackage{framed}
\definecolor{shadecolor}{RGB}{248,248,248}
\newenvironment{Shaded}{\begin{snugshade}}{\end{snugshade}}
\newcommand{\KeywordTok}[1]{\textcolor[rgb]{0.13,0.29,0.53}{\textbf{#1}}}
\newcommand{\DataTypeTok}[1]{\textcolor[rgb]{0.13,0.29,0.53}{#1}}
\newcommand{\DecValTok}[1]{\textcolor[rgb]{0.00,0.00,0.81}{#1}}
\newcommand{\BaseNTok}[1]{\textcolor[rgb]{0.00,0.00,0.81}{#1}}
\newcommand{\FloatTok}[1]{\textcolor[rgb]{0.00,0.00,0.81}{#1}}
\newcommand{\ConstantTok}[1]{\textcolor[rgb]{0.00,0.00,0.00}{#1}}
\newcommand{\CharTok}[1]{\textcolor[rgb]{0.31,0.60,0.02}{#1}}
\newcommand{\SpecialCharTok}[1]{\textcolor[rgb]{0.00,0.00,0.00}{#1}}
\newcommand{\StringTok}[1]{\textcolor[rgb]{0.31,0.60,0.02}{#1}}
\newcommand{\VerbatimStringTok}[1]{\textcolor[rgb]{0.31,0.60,0.02}{#1}}
\newcommand{\SpecialStringTok}[1]{\textcolor[rgb]{0.31,0.60,0.02}{#1}}
\newcommand{\ImportTok}[1]{#1}
\newcommand{\CommentTok}[1]{\textcolor[rgb]{0.56,0.35,0.01}{\textit{#1}}}
\newcommand{\DocumentationTok}[1]{\textcolor[rgb]{0.56,0.35,0.01}{\textbf{\textit{#1}}}}
\newcommand{\AnnotationTok}[1]{\textcolor[rgb]{0.56,0.35,0.01}{\textbf{\textit{#1}}}}
\newcommand{\CommentVarTok}[1]{\textcolor[rgb]{0.56,0.35,0.01}{\textbf{\textit{#1}}}}
\newcommand{\OtherTok}[1]{\textcolor[rgb]{0.56,0.35,0.01}{#1}}
\newcommand{\FunctionTok}[1]{\textcolor[rgb]{0.00,0.00,0.00}{#1}}
\newcommand{\VariableTok}[1]{\textcolor[rgb]{0.00,0.00,0.00}{#1}}
\newcommand{\ControlFlowTok}[1]{\textcolor[rgb]{0.13,0.29,0.53}{\textbf{#1}}}
\newcommand{\OperatorTok}[1]{\textcolor[rgb]{0.81,0.36,0.00}{\textbf{#1}}}
\newcommand{\BuiltInTok}[1]{#1}
\newcommand{\ExtensionTok}[1]{#1}
\newcommand{\PreprocessorTok}[1]{\textcolor[rgb]{0.56,0.35,0.01}{\textit{#1}}}
\newcommand{\AttributeTok}[1]{\textcolor[rgb]{0.77,0.63,0.00}{#1}}
\newcommand{\RegionMarkerTok}[1]{#1}
\newcommand{\InformationTok}[1]{\textcolor[rgb]{0.56,0.35,0.01}{\textbf{\textit{#1}}}}
\newcommand{\WarningTok}[1]{\textcolor[rgb]{0.56,0.35,0.01}{\textbf{\textit{#1}}}}
\newcommand{\AlertTok}[1]{\textcolor[rgb]{0.94,0.16,0.16}{#1}}
\newcommand{\ErrorTok}[1]{\textcolor[rgb]{0.64,0.00,0.00}{\textbf{#1}}}
\newcommand{\NormalTok}[1]{#1}




\usepackage{forarray}
\usepackage{xstring}
\newcommand{\getIndex}[2]{
  \ForEach{,}{\IfEq{#1}{\thislevelitem}{\number\thislevelcount\ExitForEach}{}}{#2}
}

\setcounter{secnumdepth}{0}

\newcommand{\getAff}[1]{
  \getIndex{#1}{Smith College}
}

\providecommand{\tightlist}{%
  \setlength{\itemsep}{0pt}\setlength{\parskip}{0pt}}

\begin{document}
\vspace*{0.2in}

% Title must be 250 characters or less.
\begin{flushleft}
{\Large
\textbf\newline{Classification of Wildtype and Mutant Zebrafish Brains via Computational
Method} % Please use "sentence case" for title and headings (capitalize only the first word in a title (or heading), the first word in a subtitle (or subheading), and any proper nouns).
}
\newline
\\
Shuli Hu\textsuperscript{\getAff{Smith College}},
Wencong Li\textsuperscript{\getAff{Smith College}},
Dejia Tang\textsuperscript{\getAff{Smith College}},
Ji Young Yun\textsuperscript{\getAff{Smith College}}\\
\bigskip
\textbf{\getAff{Smith College}}Statistical and Data Sciences, Northampton, MA\\
\bigskip
\end{flushleft}
% Please keep the abstract below 300 words
\section*{Abstract}
Classification of biological creatures' phenotypes has long been a field
that scientists study at. In this project, we utilize support vector
machine to distinguish structures of Zebrafish's brains by using data
generated from landmark analysis (cited Morgan's paper). We create a
tool for biologists to intuitively classify three-dimensional biological
shapes into two groups, usually defined as wild type and mutant, and
understand which part of the shapes have the most impact on the
classification result. This project derives from Professor Barresi's
biological image analysis research at Smith College.

% Please keep the Author Summary between 150 and 200 words
% Use first person. PLOS ONE authors please skip this step. 
% Author Summary not valid for PLOS ONE submissions.   

\linenumbers

% Use "Eq" instead of "Equation" for equation citations.
\begin{verbatim}
## Warning: package 'reticulate' was built under R version 3.4.4
\end{verbatim}

\section{Introduction}\label{introduction}

This project derives from Professor Barresi's biological image analysis
research at Smith College and provides a tool to classify the structures
within zebrafish brains via support vector machine. Our goal is to
distinguish the wild and mutant types of zebrafish brain's structures.
Morgan, a student in Barresi Lab, used landmarks analysis to divide the
points in the three-dimentional images into small wedges and computed
the landmark, which is the most representative point, within each wedge.
The image of signals in a Zebrafish brain is shown in Figure 1. The
shape is divided into 30 slices, and each slice is further divided into
8 wedges. The landmark in each wedge is calculated by taking the median
distance of all points in each wedge, \(R\). We use number of points in
each wedge and median R to run SVM models to do classifications.

\subsection{Landmark Analysis}\label{landmark-analysis}

\subsection{Programming languages
used}\label{programming-languages-used}

\subsubsection{Python}\label{python}

\subsubsection{R}\label{r}

\subsubsection{Git}\label{git}

\section{Literature Review}\label{literature-review}

Research in developmental biology has relied on the analysis of
morphological phenotypes through qualitative examination of maximum
intensity projections that surrender the power of three dimensional
data. Statistical methods to analyze visual data are needed,
particularly to detect subtle phenotypes.

Morgan et al. (2018) have utilized the open source program, Ilastik,
which employs a training based machine learning, to eliminate the image
noise. Then they preformed principal component analysis to align
commissures between samples, reducing misalignment artifacts, and
implemented a cylindrical coordinate system which preserves image
dimensionality normally lost in maximum intensity projection (MIP),
which facilitates presentation of the data, but sacrifices much of the
complexity and relational data contained in the image. Then they reduced
the points identified by the program as belonging to the structure to a
set of landmark points that describe the shape and distribution of
signal corresponding to the structure. Finally, using the landmark
system, we are able to identify and quantify structural differences and
changes in signal distribution between wild type and mutant commissures.

Landmarks describe a shape by locating a finite number of points on each
specimen. There are three basic types of landmarks: scientific,
mathematic and pseudo-landmarks. A scientific landmark is a point
assigned by an expert that corresponds between objects in some
scientifically meaningful way, for example the corner of an eye.
Mathematical landmarks are points located on an object according to some
mathematical or geometrical property of the figure. Since it does not
assume a preference of one location to another, it is particularly
useful in automated morphological recognition and analysis for
under-studied structure. Pseudo-landmarks are constructed points on an
object, located either around the outline or in between scientific or
mathematic landmarks. It is often used to approximate continuous curves
(Dryden and Mardia, 2016). This research has chosen to calculate an
automatic set of landmarks distributed across the structure in order to
avoid introducing bias due to expectations about where biological
differences should emerge.

Morgan et al. used Random Forest machine leaning method to classify the
landmarks. Although the classification is quite accurate, it is
difficult to interpret the result from biological aspects. Instead of
doing classification on all of the landmarks at the same time, we
decided to do classifacation on one landmark at a time via Support
Vector Machine. The SVM algorithm is a classification algorithm that
provides state-of-the-art performance in a wide variety of application
domains, image classification. During the past few years, SVM has been
applied very broadly within the field of computational biology
especially in pattern recognition problems, including protein remote
homology detection, microarray gene expressions analysis, prediction of
protein-protein interactions, etc.

In 1999, Jaakkola et al. ushered in stage 4 of the development of
homology detection algorithms with a paper that garnered the ``Best
paper'' award at the annual Intelligent Systems for Molecular Biology
conference. Their primary insight was that additional accuracy can be
obtained by modeling the difference between positive and negative
examples. Because the homology task required discriminating between
related and unrelated sequences, explicitly modeling the difference
between these two sets of sequences yields an extremely powerful method.

\section{Data and Variables}\label{data-and-variables}

We have 43 wildtypes samples and 35 mutant samples for training and
testing. There are 152 landmarks for each sample, with each of them
containing the following variables: number of points in that wedge
median R (micro-meter): the median of the distances to the center of the
slice of all the points in that wedge. alpha (micro-meter): distance
from the center of the landmark to the midline theta (radian): the
degree that shows the location of the wedge of a slice

We used the number of points and the median R to do classification via
support vector machine. For missing `median R' values due to absence of
points in particular landmarks, we filled them with the median value of
all the points in that landmark.

\subsection{Tidy Data}\label{tidy-data}

The original landmarks data is a wide table containing the sample index
and all the columns holding information regarding the minimum and
maximum values of Alpha and Theta, number of points, median r value, and
the type of sample for a particular sample in each landmark. However,
because all of such variables were joined by underscores in the variable
names, such as \texttt{-14.29\_-4.76\_-0.79\_0.0\_50\_pts} or
\texttt{-14.29\_-4.76\_-0.79\_0.0\_50\_r} and the value in each cell
refers to the median r value or number of points, it was very difficult
to see what each column actually represented. The ideal format of the
data set was to have the sample index, minimum and maximum Alpha,
minimum and maximum Theta, number of points, median r, and type of
sample each be its own column.

Hence, three key functions were used from the tidyr package: gather,
separate, and spread. The gather function separated the dataset into key
and value pairs for each index. The key was the column name containing
all essential information connected by underscores and the value
included the number of points or median r value. Then, the separate
function separated the result from the gather function divided the
column connected by underscore into 5 different columns, named as
\texttt{min\_alpha}, \texttt{max\_alpha}, \texttt{min\_theta},
\texttt{max\_theta}, and \texttt{ptsOrR}. This was added to the result
of the gather function that contained the index and value of each cell,
either median R or number of points. Afterwards, the spread function
widened the already wide table by expanding the \texttt{ptsOrR} column
by creating two columns, each column representing median R and the
number of points.

\subsection{Dealing with Missing
Value}\label{dealing-with-missing-value}

Samples with missing values are eliminated by Supporting vector machine.
For wedges that do not have any point in it, \texttt{median\ r} cannot
be calculated, which means that these sample will be eliminated when
running SVM. Wedges without points have biologically meanings, and we
should not ignore these wedges in our model. In order to keep the wedges
in our model, we need to artificially pick a \texttt{median\ r} value to
replace the missing ones. Supporting vector machine is sensitive to
outliers, so we cannot pick an \texttt{r} value that could become
outliers. We decided to calculate the mean of \texttt{median\ r} for the
nth landmark of all 78 samples, and then we replace the missing
\texttt{median\ r} values with the mean.

\section{Supporting Vector Machine}\label{supporting-vector-machine}

SVM's have been proven to be a powerful algorithm for supervised
clustering. During the past few years, SVM has been applied very broadly
within the field of computational biology especially in pattern
recognition problems. The goal of SVM is to find a separation line
\(f(x) = (\beta_0 + \beta_1 * x_1 + \beta_2 * x_2)\) that separates the
nearest data as clean as possible. The parameters \(\beta\) are found by
solving the optimization problem --to maximize M subject to some
restrictions -- in 2 dimensions below.

\begin{equation}
  \left.
  \begin{array}{l@{\,}l}
     \sum_{i=1}^n \beta_i^2 = 0 \\
     \ y * ( \beta_0 + \beta_1 * x_1 + \beta_2 * x_2 ) \geq M(1-\varepsilon_i) \\
     \ \varepsilon_i \geq 0 \\
     \ \sum_{i=1}^n \varepsilon_i \leq C \\
  \end{array}
  \right.
\end{equation}

\begin{itemize}
\tightlist
\item
  \(C\): tuning parameter, toleration of violation.
\item
  \(M\): margin, distance of the closest points to the hyperplane.
\item
  \(\varepsilon_i\) : slack variable, an observation is classified at
  the correct/incorrect side of the margin.
\end{itemize}

The function of the separation line:

\[f(x) = \beta_0 + \beta_1 * x_1 + \beta_2 * x_2 \]

if \(f(x) = 0\), the observation is on the seperation line.

\[ y * ( \beta_0 + \beta_1 * x_1 + \beta_2 * x_2 )\] The above is the
perpendicular distance from the ith observation to the separation line.
If it's \textgreater{}0, the observation falls at the right side of the
separation line and vice versa.

\section{Workflow}\label{workflow}

\begin{figure}
\centering
\includegraphics{zebrafish_paper_files/figure-latex/workflow-1.pdf}
\caption{Summary of Workflow}
\end{figure}

\section{User Interface}\label{user-interface}

We created an User Interface which allows users to simply input a data
file and get an visualization of the modeling result. There are two main
components in the User Interface:

\subsection{Step One: Data Processing and
Modelling}\label{step-one-data-processing-and-modelling}

This step is implemented using \textbf{Python} (version 3) and packages
including \texttt{pandas}, \texttt{nump} and \texttt{sklearn} are
required. Users would need to run and interact with the Python script
\texttt{svm.py} to pre-process the data and build the model.

The script \texttt{svm.py} contains two components: a general-purpose
\texttt{svm\_classification()} function that builds a SVM model to
classify points for a perticular landmark and a \texttt{main()} function
that runs the \texttt{svm\_classification()} function for each landmark.

\subsubsection{User Interaction}\label{user-interaction}

\begin{figure}
\includegraphics[width=4.85in]{figures/Figure2} \caption{Example of User Interaction in Step One of the User Interface}\label{fig:useri}
\end{figure}

As shown in \texttt{Figure2}, several user inputs are tooken from users
when they run the python script.

\subsubsection{Input File}\label{input-file}

Input file must contain landmark data. Variables that are needed for
classification are required to be included in the input file. In our
analysis, we used number of points in each sub-section corresponding to
each landmark of the 3D shape and the \texttt{median\ R} of points in
each wedge.

\subsubsection{Sample input file}\label{sample-input-file}

\begin{figure}
\includegraphics[width=5.11in]{figures/Figure3} \caption{Sample Data Input File of First Step of the User Interface}\label{fig:inputdata}
\end{figure}

\subsubsection{Output File}\label{output-file}

\begin{figure}
\includegraphics[width=5.65in]{figures/Figure4} \caption{Sample Data Output File of First Step of the User Interface}\label{fig:outputdata}
\end{figure}

\subsection{Step Two: Visualization Output from Step One -- Shiny
App}\label{step-two-visualization-output-from-step-one-shiny-app}

After building SVM models in step one, we insert the output from the SVM
models into step two to visualize the results. Steps two uses the
accuracy scores output from step one to create a user-friendly app which
generates visualizations to help users to understand the SVM results.

\subsubsection{Input}\label{input}

\paragraph{Data File}\label{data-file}

Input \texttt{CSV} data file must be stored in a folder called
\texttt{data} under your working directory, and the \texttt{CSV} file
must be named as \texttt{output\_data.csv}. If you do not know what your
working directory is, you can check it by using the function
\texttt{getwd()} in base R.

\paragraph{Input Variables}\label{input-variables}

All SVM models from step one produce the following 9 accuracy
measurements:

\begin{itemize}
\tightlist
\item
  Precision score of type 0
\item
  Recall score of type 0
\item
  F1 score of type 0
\item
  Precision score of type 1
\item
  Recall score of type 1
\item
  F1 score of type 1
\item
  Overall precision score
\item
  Overall recall score
\item
  Overall F1 score
\end{itemize}

These 9 accuracy scores are the variables needed in the second step of
the user interface to create the visualizations. Users can select
\texttt{channel} and \texttt{sample\ index} to filter the input dataset
to only keep the accuracy scores of the observations that users are
interested in.

In addition, users can set the threshold of the following variables:

\begin{itemize}
\tightlist
\item
  Overall precision score
\item
  Overall recall score
\item
  Overall F1 score
\end{itemize}

The dataset used to create the visualizations is rendered everytime
users cahnge one or multiple thresholds. Our app filters out the
observations that do not fulfill the threshold requirements and uses the
resulting dataset to update the histograms and heatmaps.

\subsubsection{Output}\label{output}

We visualize the 9 accuracy scores by using both histograms and the
corresponding heatmaps that display the scores inlcuded in the
histograms in rectangular shapes that are colored with different shades
of blue according to their magnitudes. The positions of the shapes are
determined with respect to their relative positions within the
biological structure. In the study of Zebrafish, we used the relative
positions of the wedges used in landmark analysis to determine the
position of the wedges in the heatmap.

There are 10 tabs included in the user interface of the app: 1 Accuracy
Threshold Summary tab and 9 accuracy score visualization tabs.

\begin{figure}

{\centering \includegraphics[width=5.54in]{figures/shiny1} 

}

\caption{User Interface: Accuracy Threshold Summary Tab of AT Channel}\label{fig:shiny1}
\end{figure}

Figure 5 displays the Accuracy Score Threshold Summary tab of the first
sample of AT channel. Users can drag the dot on the slidebar to set the
thresholds of overall precision, recall, and f1 scores. The threshold of
the three scores are updated in the summary table. Default thresholds
are 0 for all three accuracy measurements. We then use the landmark
observations that fulfill the threshold requirements to predict the type
of the sample of choice by doing a majority vote. We simply count the
total number of landmarks that are classified as type 0 and type 1, and
then we determine whether there are more of them that are classified as
type 0 or type 1. The type that gets more vote is the predicted type of
the sample. The resuting predicted sample type is also updated in the
summary table.

Other information, such as the true type of the sample and the number of
wildetypes and mutants used in training the SVM models are also included
in the summary table.

\begin{figure}

{\centering \includegraphics[width=5.54in]{figures/shiny2} 

}

\caption{User Interface: Precision Score Visualization Tab of AT Channel}\label{fig:shiny2}
\end{figure}

Figure 6 displays the Precision Score Visualization tab of the first
sample of AT channel. In this case, all three thresholds are at default
level, 0. Therefore, all landmarks' precision scores are shown in both
the histogram and the heatmap.

\begin{figure}

{\centering \includegraphics[width=5.54in]{figures/shiny3} 

}

\caption{User Interface: Precision Score Visualization Tab of AT Channel, with precision threshold = 0.75}\label{fig:shiny3}
\end{figure}

Figure 7 also displays the Precision Score Visualization tab of the
first sample of AT channel. In this case, recall and f1 scores'
thresholds are at default level and precision threshold is set to be
0.75. Therefore, only landmarks that have precision scores that are
equal to or greater than 0.75 are shown in the visualizations. As shown
in the histogram, all values less than 0.75 are removed from the
histogram in figure 6. Some of the blocks in figure 6 are turned into
blank blocks after the precision threshold is increased to 0.75.

\begin{figure}

{\centering \includegraphics[width=5.54in]{figures/shiny4} 

}

\caption{User Interface: Recall Score Visualization Tab of AT Channel}\label{fig:shiny4}
\end{figure}

Figure 8 displays the Recall Score Visualization tab of the first sample
of AT channel.

\begin{figure}

{\centering \includegraphics[width=5.54in]{figures/shiny5} 

}

\caption{User Interface: Recall Score Visualization Tab of AT Channel, with recall threshold = 0.75}\label{fig:shiny5}
\end{figure}

Figure 9 displays the Recall Score Visualization tab of the first sample
of AT channel with recall threshold equals to 0.75.

\begin{figure}

{\centering \includegraphics[width=5.54in]{figures/shiny6} 

}

\caption{User Interface: F1 Score Visualization Tab of AT Channel}\label{fig:shiny6}
\end{figure}

Figure 10 displays the F1 Score Visualization tab of the first sample of
AT channel.

\begin{figure}

{\centering \includegraphics[width=5.54in]{figures/shiny7} 

}

\caption{User Interface: F1 Score Visualization Tab of AT Channel, with f1 threshold = 0.75}\label{fig:shiny7}
\end{figure}

Figure 11 displays the F1 Score Visualization tab of the first sample of
AT channel with f1 threshold equals to 0.75.

\begin{figure}

{\centering \includegraphics[width=5.54in]{figures/shiny9} 

}

\caption{User Interface: F1 Score Visualization Tab of ZRF Channel}\label{fig:shiny9}
\end{figure}

Users can also choose to observe the SVM results of ZRF channel. Figure
12 displays the F1 Score Visualization tab of the first sample of ZRF
channel with all thresholds equal to 0.

\section{Analyzing Result}\label{analyzing-result}

\subsection{Testing}\label{testing}

\subsubsection{Cross-Validation}\label{cross-validation}

For our project, we have access to 43 wild-type samples and 35
mutant-type samples. Due to this limited sample size, we dicided to use
a leave-one-out cross validation method to test our model.\\
For each testing sample, we built 152 SVMs for each landmark. For each
SVM, we used 10-fold cross validation to select a tuning parameter C
value among 0.1, 1 and 10.

\subsubsection{Processing Results and Make
Predictions}\label{processing-results-and-make-predictions}

After we get precision scores and predictions of each landmark, we will
present the distribution of the landmarks' precision scores. The user
would then be allowed to set a threshold for certain precision scores to
select out a subset of landmarks that are considered significant. A
majority vote would then be performed among the selected landmarks to
get an overall prediction for the sample.

\section{Result}\label{result}

\subsection{Prediction}\label{prediction}

\subsection{Visulization}\label{visulization}

\section{Discussion}\label{discussion}

\subsection{Strengths}\label{strengths}

In the previous method random forest, the number of predictors p exceeds
the number of samples. Morgan applied PCA do reduce the dimension of the
predictors. The problem with dimension reduction is that it gives a
linear combination of the dimensions that are projected on those are
kept. While the largest projections still make sense, the minor
projections are very random and thus difficult to interpret.

The SVM model generated based on landmark data gives insightful analysis
of: 1. which landmark, or which part of the Zebrafish brain, has more
predictive power 2. whether a new Zebrafish brain sample is a mutant or
wild type

\subsection{Limitations}\label{limitations}

The SVM model also has its limitation in that it only considers one
single landmark at a time without considering the relationship across
the landmarks of the whole sample.

\subsection{Improvements}\label{improvements}

\subsubsection{Itenerating machine
learning}\label{itenerating-machine-learning}

Instead of cross-validation, better results could be achieved by using
itenerating machine learning method. In iterative machine learning we
repeat the process of training and testing several times. At the first
round the user gives examples of objects belonging to some classes and
the machine learning algorithm is trained with this data. In the second
round, the algorithm shows examples of objects it thinks that belong to
these classes. Now, the user merely adds objects to the improved
training set which the machine learning algorithm has put into a wrong
class. That is, the user only corrects the ``misunderstandings'' of the
algorithm. In this way we can concentrate on difficult examples of
objects that are hard to classify or are for some reason easily missed
by humans. Such objects may lie close to the decision boundaries or in
the periphery in the multidimensional feature space. This iterative
process is continued until the machine learning algorithm does not make
any mistakes or the classification results do not improve anymore. It
will improve our classification results and thus is likely to help make
better predictions for unknown type.

\subsection{Future Study}\label{future-study}

\begin{enumerate}
\def\labelenumi{\arabic{enumi}.}
\tightlist
\item
  making the program more user friendly
\item
  running more tests to prove the accuracy of our model
\end{enumerate}

\section{Acknowledgements}\label{acknowledgements}

This project was completed in partial fulfillment of the requirements of
SDS 410: SDS Capstone. This course is offered by the Statistical and
Data Sciences Program at Smith College, and was taught by Benjamin
Baumer in Spring 2018.

\section{Appendix Code}\label{appendix-code}

\subsection{Support Vector Machine}\label{support-vector-machine}

\begin{Shaded}
\begin{Highlighting}[]
\ImportTok{import}\NormalTok{ pandas }\ImportTok{as}\NormalTok{ pd}
\ImportTok{import}\NormalTok{ numpy }\ImportTok{as}\NormalTok{ np}
\ImportTok{from}\NormalTok{ sklearn.metrics }\ImportTok{import}\NormalTok{ confusion_matrix, classification_report}
\ImportTok{from}\NormalTok{ sklearn.model_selection }\ImportTok{import}\NormalTok{ GridSearchCV}
\ImportTok{from}\NormalTok{ sklearn.svm }\ImportTok{import}\NormalTok{ SVC}
\ImportTok{from}\NormalTok{ sklearn.metrics }\ImportTok{import}\NormalTok{ f1_score, precision_score, recall_score}
\CommentTok{'''}
\CommentTok{A function that builds a SVM model with linear kernel to classify points to two classes.}
\CommentTok{Inputs:}
\CommentTok{training_landmarks - a pandas dataframe containing all training landmark data.}
\CommentTok{index              - a perticular landmark id of interest. eg. '101'}
\CommentTok{x_names            - a list of explanatory variable names. eg. ['pts', 'r']}
\CommentTok{y_name             - a string representing response variable name. eg. 'stype'}
\CommentTok{class0             - name of the first class. eg. 'wt-at'}
\CommentTok{class1             - name of the second class. eg. 'mt-at'}
\CommentTok{C_values           - a list of tunning variable C (penalty parameter of the error term) that the method would grid-search on. Default value is [0.1, 1, 10].}
\CommentTok{Output:}
\CommentTok{svm                - the SVM model trained from the training dataset}
\CommentTok{ww                 - among the training samples, the number of wild type samples with chosen landmark predicted as wild type.}
\CommentTok{wm                 - among the training samples, the number of wild type samples with chosen landmark predicted as mutant type.}
\CommentTok{mm                 - among the training samples, the number of mutant type samples with chosen landmark predicted as mutant type.}
\CommentTok{mw                 - among the training samples, the number of mutant type samples with chosen landmark predicted as wild type.}
\CommentTok{'''}
\KeywordTok{def}\NormalTok{ svm_classification(training_landmarks, index, x_names, y_name, class0, class1, C_values }\OperatorTok{=}\NormalTok{ [}\FloatTok{0.1}\NormalTok{, }\DecValTok{1}\NormalTok{, }\DecValTok{10}\NormalTok{] ):}
    \CommentTok{# filter out the landmarks needed}
\NormalTok{    chosenLandmark }\OperatorTok{=}\NormalTok{ landmarks[landmarks.landmark_index}\OperatorTok{==}\NormalTok{index]}
\NormalTok{    chosenLandmark }\OperatorTok{=}\NormalTok{ chosenLandmark[np.isfinite(chosenLandmark[}\StringTok{'r'}\NormalTok{])]}
    
    \CommentTok{# create training and testing data}
\NormalTok{    X }\OperatorTok{=}\NormalTok{ chosenLandmark[x_names]}
\NormalTok{    y }\OperatorTok{=}\NormalTok{ chosenLandmark[y_name]}
\NormalTok{    y }\OperatorTok{=}\NormalTok{ y.replace([class1], }\DecValTok{1}\NormalTok{)}
\NormalTok{    y }\OperatorTok{=}\NormalTok{ y.replace([class0], }\DecValTok{0}\NormalTok{)}
    \CommentTok{# check whether both classes exist}
\NormalTok{    count_1 }\OperatorTok{=}\NormalTok{ chosenLandmark[y_name].}\BuiltInTok{str}\NormalTok{.contains(class1).}\BuiltInTok{sum}\NormalTok{()}
\NormalTok{    count_0 }\OperatorTok{=}\NormalTok{ chosenLandmark[y_name].}\BuiltInTok{str}\NormalTok{.contains(class0).}\BuiltInTok{sum}\NormalTok{()}
    \ControlFlowTok{if}\NormalTok{ (count_1 }\OperatorTok{<} \DecValTok{2} \KeywordTok{or}\NormalTok{ count_0 }\OperatorTok{<} \DecValTok{2}\NormalTok{):}
        \ControlFlowTok{return} \VariableTok{None}\NormalTok{, }\VariableTok{None}\NormalTok{, }\VariableTok{None}\NormalTok{, }\VariableTok{None}\NormalTok{, }\VariableTok{None}
    \CommentTok{# find the best C value by cross-validation}
\NormalTok{    tuned_parameters }\OperatorTok{=}\NormalTok{ [\{}\StringTok{'C'}\NormalTok{: C_values\}]}
\NormalTok{    clf }\OperatorTok{=}\NormalTok{ GridSearchCV(SVC(kernel}\OperatorTok{=}\StringTok{'linear'}\NormalTok{), tuned_parameters, cv}\OperatorTok{=}\DecValTok{10}\NormalTok{, scoring}\OperatorTok{=}\StringTok{'accuracy'}\NormalTok{)}
\NormalTok{    clf.fit(X.values, y.values)}
\NormalTok{    best_c }\OperatorTok{=}\NormalTok{ clf.best_params_[}\StringTok{'C'}\NormalTok{]}
    
\NormalTok{    svc }\OperatorTok{=}\NormalTok{ SVC(C}\OperatorTok{=}\NormalTok{best_c, kernel}\OperatorTok{=}\StringTok{'linear'}\NormalTok{)}
\NormalTok{    svc.fit(X, y)}
    
\NormalTok{    prediction }\OperatorTok{=}\NormalTok{ svc.predict(X)}
    \CommentTok{# print confusion matrix}
    \BuiltInTok{print}\NormalTok{(}\StringTok{"confusion matrix: "}\NormalTok{)}
\NormalTok{    cm }\OperatorTok{=}\NormalTok{ confusion_matrix(y, prediction)}
\NormalTok{    cm_df }\OperatorTok{=}\NormalTok{ pd.DataFrame(cm.T, index}\OperatorTok{=}\NormalTok{svc.classes_, columns}\OperatorTok{=}\NormalTok{svc.classes_)}
    \BuiltInTok{print}\NormalTok{(cm_df)}
    \CommentTok{# Statistics of training precision:}
    \CommentTok{# number of wild type samples with this landmark predicted as wild type.}
\NormalTok{    ww }\OperatorTok{=}\DecValTok{0}
    \CommentTok{# number of wild type samples with this landmark predicted as mutant type.}
\NormalTok{    wm }\OperatorTok{=} \DecValTok{0}
    \CommentTok{# number of mutant type samples with this landmark predicted as mutant type.}
\NormalTok{    mm }\OperatorTok{=} \DecValTok{0}
    \CommentTok{# number of mutant type samples with this landmark predicted as wild type.}
\NormalTok{    mw }\OperatorTok{=} \DecValTok{0}
    
    \ControlFlowTok{for}\NormalTok{ i }\KeywordTok{in} \BuiltInTok{range}\NormalTok{ (}\BuiltInTok{len}\NormalTok{(y)):}
\NormalTok{        _y }\OperatorTok{=}\NormalTok{ y.values[i]}
\NormalTok{        _p }\OperatorTok{=}\NormalTok{ prediction[i]}
        \ControlFlowTok{if}\NormalTok{ _y}\OperatorTok{==}\DecValTok{1} \KeywordTok{and}\NormalTok{ _p}\OperatorTok{==}\DecValTok{1}\NormalTok{:}
\NormalTok{            mm }\OperatorTok{=}\NormalTok{ mm }\OperatorTok{+} \DecValTok{1}
        \ControlFlowTok{elif}\NormalTok{ _y}\OperatorTok{==}\DecValTok{1} \KeywordTok{and}\NormalTok{ _p}\OperatorTok{==}\DecValTok{0}\NormalTok{:}
\NormalTok{            mw }\OperatorTok{=}\NormalTok{ mw }\OperatorTok{+} \DecValTok{1}
        \ControlFlowTok{elif}\NormalTok{ _y}\OperatorTok{==}\DecValTok{0} \KeywordTok{and}\NormalTok{ _p}\OperatorTok{==}\DecValTok{0}\NormalTok{:}
\NormalTok{            ww }\OperatorTok{=}\NormalTok{ ww }\OperatorTok{+} \DecValTok{1}
        \ControlFlowTok{elif}\NormalTok{ _y}\OperatorTok{==}\DecValTok{0} \KeywordTok{and}\NormalTok{ _p}\OperatorTok{==}\DecValTok{1}\NormalTok{:}
\NormalTok{            wm }\OperatorTok{=}\NormalTok{ wm }\OperatorTok{+} \DecValTok{1}
    
    \ControlFlowTok{return}\NormalTok{ svc, ww, wm, mm, mw}
\ControlFlowTok{if} \VariableTok{__name__} \OperatorTok{==} \StringTok{"__main__"}\NormalTok{:}
    \CommentTok{# Get interested chnnel name}
\NormalTok{    channel }\OperatorTok{=} \StringTok{''}
    \ControlFlowTok{while}\NormalTok{ (channel }\OperatorTok{!=} \StringTok{'AT'} \KeywordTok{and}\NormalTok{ channel }\OperatorTok{!=} \StringTok{'ZRF'}\NormalTok{):}
\NormalTok{        channel }\OperatorTok{=} \BuiltInTok{input}\NormalTok{(}\StringTok{"Please enter 'AT' or 'ZRF' to indicate channel interested: "}\NormalTok{)}
    
\NormalTok{    class0 }\OperatorTok{=} \StringTok{'mt-zrf'} \ControlFlowTok{if}\NormalTok{ channel }\OperatorTok{==} \StringTok{'ZRF'} \ControlFlowTok{else} \StringTok{'mt-at'}
\NormalTok{    class1 }\OperatorTok{=} \StringTok{'wt-zrf'} \ControlFlowTok{if}\NormalTok{ channel }\OperatorTok{==} \StringTok{'ZRF'} \ControlFlowTok{else} \StringTok{'wt-at'}
    \CommentTok{# Read in landmark data}
\NormalTok{    data_type }\OperatorTok{=} \StringTok{'-1'}
    \ControlFlowTok{while}\NormalTok{ (data_type }\OperatorTok{!=} \StringTok{'0'} \KeywordTok{and}\NormalTok{ data_type }\OperatorTok{!=} \StringTok{'1'}\NormalTok{):}
\NormalTok{        data_type }\OperatorTok{=} \BuiltInTok{input}\NormalTok{(}\StringTok{"Enter 0 for filling NaN values with median and 1 for filling with 2*median: "}\NormalTok{)}
\NormalTok{    landmarks }\OperatorTok{=}\NormalTok{ pd.DataFrame()}
    \ControlFlowTok{if}\NormalTok{ (channel }\OperatorTok{==} \StringTok{'AT'}\NormalTok{):}
\NormalTok{        landmarks }\OperatorTok{=}\NormalTok{ pd.read_csv(}\StringTok{'./data/final/landmark_AT_filled_w_median.csv'}\NormalTok{) }\ControlFlowTok{if}\NormalTok{ data_type}\OperatorTok{==}\StringTok{'0'} \ControlFlowTok{else}\NormalTok{ pd.read_csv(}\StringTok{'./data/final/landmark_AT_filled_w_2median.csv'}\NormalTok{)}
    \ControlFlowTok{else}\NormalTok{:}
\NormalTok{        landmarks }\OperatorTok{=}\NormalTok{ pd.read_csv(}\StringTok{'./data/final/landmark_ZRF_filled_w_median.csv'}\NormalTok{) }\ControlFlowTok{if}\NormalTok{ data_type}\OperatorTok{==}\StringTok{'0'} \ControlFlowTok{else}\NormalTok{ pd.read_csv(}\StringTok{'./data/final/landmark_ZRF_filled_w_2median.csv'}\NormalTok{)}
    \CommentTok{# Get sample id}
\NormalTok{    sample }\OperatorTok{=}\NormalTok{ pd.DataFrame()}
    \ControlFlowTok{while}\NormalTok{(sample.shape[}\DecValTok{0}\NormalTok{]}\OperatorTok{<}\DecValTok{2}\NormalTok{):}
\NormalTok{        sample_id }\OperatorTok{=} \BuiltInTok{str}\NormalTok{(}\BuiltInTok{input}\NormalTok{(}\StringTok{"Please enter a VALID sample index: "}\NormalTok{))}
\NormalTok{        sample }\OperatorTok{=}\NormalTok{ landmarks[landmarks.sample_index}\OperatorTok{==}\NormalTok{sample_id]}
    \CommentTok{# Get result file's name and create the file with column names}
\NormalTok{    result_file_name }\OperatorTok{=} \BuiltInTok{str}\NormalTok{(}\BuiltInTok{input}\NormalTok{(}\StringTok{"Please enter result file name: "}\NormalTok{))}
\NormalTok{    result_file }\OperatorTok{=} \BuiltInTok{open}\NormalTok{(result_file_name, }\StringTok{'w'}\NormalTok{)}
\NormalTok{    result_file.write(}\StringTok{'sample_index, landmark_index, pred, ww, wm, mm, mw}\CharTok{\textbackslash{}n}\StringTok{'}\NormalTok{)}
\NormalTok{    result_file.close()}
    \CommentTok{# Get existing landmark ids}
\NormalTok{    landmark_ids }\OperatorTok{=}\NormalTok{ sample[}\StringTok{'landmark_index'}\NormalTok{]}
\NormalTok{    leave_one_out }\OperatorTok{=}\NormalTok{ landmarks[landmarks.sample_index}\OperatorTok{!=}\NormalTok{sample_id]}
    \ControlFlowTok{for}\NormalTok{ l }\KeywordTok{in}\NormalTok{ landmark_ids.values:}
        \BuiltInTok{print}\NormalTok{ (}\StringTok{"======================================="}\NormalTok{)}
        \BuiltInTok{print}\NormalTok{ (}\StringTok{"landmark: "}\NormalTok{, }\BuiltInTok{str}\NormalTok{(l))}
\NormalTok{        svc, ww, wm, mm, mw }\OperatorTok{=}\NormalTok{ svm_classification(training_landmarks }\OperatorTok{=}\NormalTok{ leave_one_out,}
\NormalTok{                                                 index }\OperatorTok{=}\NormalTok{ l,}
\NormalTok{                                                 x_names }\OperatorTok{=}\NormalTok{ [}\StringTok{'pts'}\NormalTok{, }\StringTok{'r'}\NormalTok{],}
\NormalTok{                                                 y_name }\OperatorTok{=} \StringTok{'stype'}\NormalTok{,}
\NormalTok{                                                 class0 }\OperatorTok{=}\NormalTok{ class0,}
\NormalTok{                                                 class1 }\OperatorTok{=}\NormalTok{ class1,}
\NormalTok{                                                 C_values }\OperatorTok{=}\NormalTok{ [}\FloatTok{0.1}\NormalTok{, }\DecValTok{1}\NormalTok{, }\DecValTok{10}\NormalTok{])}
        \ControlFlowTok{if}\NormalTok{ (svc }\KeywordTok{is} \VariableTok{None}\NormalTok{):}
            \BuiltInTok{print}\NormalTok{(}\StringTok{"One of the classes have too few samples for this landmark, so skipping it."}\NormalTok{)}
            \ControlFlowTok{continue}
\NormalTok{        prediction }\OperatorTok{=}\NormalTok{ svc.predict(sample[sample.landmark_index}\OperatorTok{==}\NormalTok{l][[}\StringTok{'pts'}\NormalTok{, }\StringTok{'r'}\NormalTok{]])}
\NormalTok{        result }\OperatorTok{=} \StringTok{', '}\NormalTok{.join(}\BuiltInTok{str}\NormalTok{(x) }\ControlFlowTok{for}\NormalTok{ x }\KeywordTok{in}\NormalTok{ [sample_id, l, prediction[}\DecValTok{0}\NormalTok{], ww, wm, mm, mw]) }\OperatorTok{+} \StringTok{'}\CharTok{\textbackslash{}n}\StringTok{'}
        \BuiltInTok{print}\NormalTok{(}\StringTok{'result:'}\NormalTok{, result)}
\NormalTok{        result_file }\OperatorTok{=} \BuiltInTok{open}\NormalTok{(result_file_name, }\StringTok{'a'}\NormalTok{)}
\NormalTok{        result_file.write(result)}
\NormalTok{        result_file.close()}
\end{Highlighting}
\end{Shaded}

\subsection{Shiny App}\label{shiny-app}

\begin{Shaded}
\begin{Highlighting}[]
\NormalTok{list_of_indices <-}\StringTok{ }\KeywordTok{c}\NormalTok{(index}\OperatorTok{$}\NormalTok{Index, }\StringTok{"AT"}\NormalTok{, }\StringTok{"ZRF"}\NormalTok{)}
\NormalTok{list_of_scores <-}\StringTok{ }\KeywordTok{c}\NormalTok{(}\StringTok{"precision"}\NormalTok{, }\StringTok{"recall"}\NormalTok{, }\StringTok{"f1"}\NormalTok{, }\StringTok{"w_precision"}\NormalTok{, }\StringTok{"w_recall"}\NormalTok{, }\StringTok{"w_f1"}\NormalTok{, }\StringTok{"m_precision"}\NormalTok{, }\StringTok{"m_recall"}\NormalTok{, }\StringTok{"m_f1"}\NormalTok{)}
\NormalTok{landmark_xy <-}\StringTok{ }\KeywordTok{fread}\NormalTok{(}\StringTok{"/Users/priscilla/Desktop/SDS Capstone/Zebrafish/analysis/landmark_xy.csv"}\NormalTok{)}
\NormalTok{list_of_channel <-}\StringTok{ }\KeywordTok{c}\NormalTok{(}\StringTok{"AT"}\NormalTok{, }\StringTok{"ZRF"}\NormalTok{)}

\CommentTok{# User Interface}
\NormalTok{ui <-}\StringTok{ }\KeywordTok{fluidPage}\NormalTok{(}
  \KeywordTok{titlePanel}\NormalTok{(}\DataTypeTok{title=}\KeywordTok{h4}\NormalTok{(}\StringTok{"Classification of Wildtype and Mutant Zebrafish Brains via Computational Method"}\NormalTok{, }
                      \DataTypeTok{align=}\StringTok{"center"}\NormalTok{)),}
  \KeywordTok{selectInput}\NormalTok{(}\StringTok{"channel"}\NormalTok{, }\StringTok{"Channel:"}\NormalTok{, list_of_channel),}
  \KeywordTok{selectInput}\NormalTok{(}\StringTok{"sampleindex"}\NormalTok{, }\StringTok{"Sample Index:"}\NormalTok{, list_of_indices),}
  \KeywordTok{selectInput}\NormalTok{(}\StringTok{"score"}\NormalTok{, }\StringTok{"Accuracy Measurement:"}\NormalTok{, list_of_scores),}
  \KeywordTok{mainPanel}\NormalTok{(}\KeywordTok{fluidRow}\NormalTok{(}
              \KeywordTok{splitLayout}\NormalTok{(}\DataTypeTok{cellWidths =} \KeywordTok{c}\NormalTok{(}\StringTok{"90%"}\NormalTok{, }\StringTok{"60%"}\NormalTok{), }\KeywordTok{plotOutput}\NormalTok{(}\StringTok{"plot1"}\NormalTok{), }\KeywordTok{plotOutput}\NormalTok{(}\StringTok{"plot2"}\NormalTok{))}
\NormalTok{            ))}
\NormalTok{)}

\CommentTok{# Server}
\NormalTok{server <-}\StringTok{ }\ControlFlowTok{function}\NormalTok{(input,output) \{}
\NormalTok{  dat <-}\StringTok{ }\KeywordTok{reactive}\NormalTok{(\{}
\NormalTok{    dir <-}\StringTok{ }\KeywordTok{paste0}\NormalTok{(wd, }\StringTok{"/analysis/r"}\NormalTok{, input}\OperatorTok{$}\NormalTok{sampleindex, }\StringTok{"_med_"}\NormalTok{, input}\OperatorTok{$}\NormalTok{channel, }\StringTok{"_result.csv"}\NormalTok{)}
\NormalTok{    test <-}\StringTok{ }\KeywordTok{fread}\NormalTok{(dir)}
\NormalTok{    test <-}\StringTok{ }\NormalTok{test }\OperatorTok
\StringTok{      }\KeywordTok{left_join}\NormalTok{(landmark_xy, }\DataTypeTok{by=}\StringTok{"landmark_index"}\NormalTok{)}
    \KeywordTok{print}\NormalTok{(test)}
\NormalTok{    test}
\NormalTok{  \})}
  
  \CommentTok{# Plot One}
\NormalTok{  output}\OperatorTok{$}\NormalTok{plot1 <-}\StringTok{ }\KeywordTok{renderPlot}\NormalTok{(\{}
\NormalTok{    p1 <-}\StringTok{ }\KeywordTok{ggplot}\NormalTok{(}\KeywordTok{dat}\NormalTok{(), }
                 \KeywordTok{aes}\NormalTok{(}\DataTypeTok{x =}\NormalTok{ y, }\DataTypeTok{y =}\NormalTok{ x)) }\OperatorTok{+}
\StringTok{      }\KeywordTok{geom_tile}\NormalTok{(}\KeywordTok{aes}\NormalTok{(}\DataTypeTok{fill =}\NormalTok{ input}\OperatorTok{$}\NormalTok{score)) }\OperatorTok{+}
\StringTok{      }\KeywordTok{xlab}\NormalTok{(}\StringTok{"Alpha"}\NormalTok{) }\OperatorTok{+}
\StringTok{      }\KeywordTok{ylab}\NormalTok{(}\StringTok{"Theta"}\NormalTok{) }\OperatorTok{+}
\StringTok{      }\KeywordTok{scale_x_continuous}\NormalTok{(}\DataTypeTok{limits =} \KeywordTok{c}\NormalTok{(}\DecValTok{1}\NormalTok{, }\DecValTok{19}\NormalTok{), }\DataTypeTok{breaks=}\KeywordTok{c}\NormalTok{(}\DecValTok{1}\NormalTok{, }\DecValTok{10}\NormalTok{, }\DecValTok{19}\NormalTok{), }\DataTypeTok{labels=}\KeywordTok{c}\NormalTok{(}\StringTok{"-90.51"}\NormalTok{, }\StringTok{"0"}\NormalTok{, }\StringTok{"90.51"}\NormalTok{)) }\OperatorTok{+}
\StringTok{      }\KeywordTok{scale_y_continuous}\NormalTok{(}\DataTypeTok{limits =} \KeywordTok{c}\NormalTok{(}\DecValTok{1}\NormalTok{, }\DecValTok{8}\NormalTok{), }\DataTypeTok{breaks=}\KeywordTok{c}\NormalTok{(}\DecValTok{1}\NormalTok{, }\FloatTok{4.5}\NormalTok{, }\DecValTok{8}\NormalTok{), }\DataTypeTok{labels=}\KeywordTok{c}\NormalTok{(}\StringTok{"-3.14"}\NormalTok{,}\StringTok{"0"}\NormalTok{,}\StringTok{"3.14"}\NormalTok{)) }\OperatorTok{+}
\StringTok{      }\KeywordTok{scale_fill_continuous}\NormalTok{(}\DataTypeTok{limits=}\KeywordTok{c}\NormalTok{(}\DecValTok{0}\NormalTok{, }\DecValTok{1}\NormalTok{), }\DataTypeTok{breaks=}\KeywordTok{seq}\NormalTok{(}\DecValTok{0}\NormalTok{,}\DecValTok{1}\NormalTok{,}\DataTypeTok{by=}\FloatTok{0.25}\NormalTok{)) }
\NormalTok{    p1}
\NormalTok{  \})}
  
  \CommentTok{# Plot Two}
\NormalTok{  output}\OperatorTok{$}\NormalTok{plot2 <-}\StringTok{ }\KeywordTok{renderPlot}\NormalTok{(\{}
\NormalTok{    p2 <-}\StringTok{ }\KeywordTok{qplot}\NormalTok{(input}\OperatorTok{$}\NormalTok{score, }\DataTypeTok{geom =} \StringTok{"histogram"}\NormalTok{) }\OperatorTok{+}
\StringTok{      }\KeywordTok{xlab}\NormalTok{(}\StringTok{"Precision"}\NormalTok{) }\OperatorTok{+}
\StringTok{      }\KeywordTok{ylab}\NormalTok{(}\StringTok{"Count"}\NormalTok{)  }
\NormalTok{    p2}
\NormalTok{  \})}
  
\NormalTok{\}}
\end{Highlighting}
\end{Shaded}

\section*{References}\label{references}
\addcontentsline{toc}{section}{References}

\nolinenumbers


\end{document}

