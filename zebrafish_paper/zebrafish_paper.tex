% Template for PLoS
% Version 3.4 January 2017
\documentclass[10pt,letterpaper]{article}
\usepackage[top=0.85in,left=2.75in,footskip=0.75in]{geometry}

% amsmath and amssymb packages, useful for mathematical formulas and symbols
\usepackage{amsmath,amssymb}

% Use adjustwidth environment to exceed column width (see example table in text)
\usepackage{changepage}

% Use Unicode characters when possible
\usepackage[utf8x]{inputenc}

% textcomp package and marvosym package for additional characters
\usepackage{textcomp,marvosym}

% cite package, to clean up citations in the main text. Do not remove.
% \usepackage{cite}

% Use nameref to cite supporting information files (see Supporting Information section for more info)
\usepackage{nameref,hyperref}

% line numbers
\usepackage[right]{lineno}

% ligatures disabled
\usepackage{microtype}
\DisableLigatures[f]{encoding = *, family = * }

% color can be used to apply background shading to table cells only
\usepackage[table]{xcolor}

% array package and thick rules for tables
\usepackage{array}

% create "+" rule type for thick vertical lines
\newcolumntype{+}{!{\vrule width 2pt}}

% create \thickcline for thick horizontal lines of variable length
\newlength\savedwidth
\newcommand\thickcline[1]{%
  \noalign{\global\savedwidth\arrayrulewidth\global\arrayrulewidth 2pt}%
  \cline{#1}%
  \noalign{\vskip\arrayrulewidth}%
  \noalign{\global\arrayrulewidth\savedwidth}%
}

% \thickhline command for thick horizontal lines that span the table
\newcommand\thickhline{\noalign{\global\savedwidth\arrayrulewidth\global\arrayrulewidth 2pt}%
\hline
\noalign{\global\arrayrulewidth\savedwidth}}


% Remove comment for double spacing
%\usepackage{setspace} 
%\doublespacing

% Text layout
\raggedright
\setlength{\parindent}{0.5cm}
\textwidth 5.25in 
\textheight 8.75in

% Bold the 'Figure #' in the caption and separate it from the title/caption with a period
% Captions will be left justified
\usepackage[aboveskip=1pt,labelfont=bf,labelsep=period,justification=raggedright,singlelinecheck=off]{caption}
\renewcommand{\figurename}{Fig}

% Use the PLoS provided BiBTeX style
% \bibliographystyle{plos2015}

% Remove brackets from numbering in List of References
\makeatletter
\renewcommand{\@biblabel}[1]{\quad#1.}
\makeatother

% Leave date blank
\date{}

% Header and Footer with logo
\usepackage{lastpage,fancyhdr,graphicx}
\usepackage{epstopdf}
\pagestyle{myheadings}
\pagestyle{fancy}
\fancyhf{}
\setlength{\headheight}{27.023pt}
\lhead{\includegraphics[width=2.0in]{PLOS-submission.eps}}
\rfoot{\thepage/\pageref{LastPage}}
\renewcommand{\footrule}{\hrule height 2pt \vspace{2mm}}
\fancyheadoffset[L]{2.25in}
\fancyfootoffset[L]{2.25in}
\lfoot{\sf PLOS}

%% Include all macros below
\newcommand{\lorem}{{\bf LOREM}}
\newcommand{\ipsum}{{\bf IPSUM}}





\usepackage{forarray}
\usepackage{xstring}
\newcommand{\getIndex}[2]{
  \ForEach{,}{\IfEq{#1}{\thislevelitem}{\number\thislevelcount\ExitForEach}{}}{#2}
}

\setcounter{secnumdepth}{0}

\newcommand{\getAff}[1]{
  \getIndex{#1}{Smith College}
}

\providecommand{\tightlist}{%
  \setlength{\itemsep}{0pt}\setlength{\parskip}{0pt}}

\begin{document}
\vspace*{0.2in}

% Title must be 250 characters or less.
\begin{flushleft}
{\Large
\textbf\newline{Classification of Wildtype and Mutant Zebrafish Brains via Computational
Method} % Please use "sentence case" for title and headings (capitalize only the first word in a title (or heading), the first word in a subtitle (or subheading), and any proper nouns).
}
\newline
\\
Shuli Hu\textsuperscript{\getAff{Smith College}},
Wencong Li\textsuperscript{\getAff{Smith College}},
Dejia Tang\textsuperscript{\getAff{Smith College}},
Ji Young Yun\textsuperscript{\getAff{Smith College}}\\
\bigskip
\textbf{\getAff{Smith College}}Statistical and Data Sciences, Northampton, MA\\
\bigskip
\end{flushleft}
% Please keep the abstract below 300 words

% Please keep the Author Summary between 150 and 200 words
% Use first person. PLOS ONE authors please skip this step. 
% Author Summary not valid for PLOS ONE submissions.   

\linenumbers

% Use "Eq" instead of "Equation" for equation citations.
\emph{Text based on plos sample manuscript, see
\url{http://journals.plos.org/ploscompbiol/s/latex}}

\section{Introduction}\label{introduction}

(PL)

\subsection{Programming languages
used}\label{programming-languages-used}

\subsubsection{Python}\label{python}

\subsubsection{R}\label{r}

\subsubsection{Git}\label{git}

\section{Literature Review}\label{literature-review}

(Shuli)

\section{Data}\label{data}

(PL) 48 Wildtypes 14 mutants

\subsection{Variables}\label{variables}

Number of points Median r Alpha Theta

\subsection{Tidy Data}\label{tidy-data}

The original landmarks data is a wide table containing the sample index
and all the columns holding information regarding the minimum and
maximum values of Alpha and Theta, number of points, median r value, and
the type of sample for a particular sample in each landmark. However,
because all of such variables were joined by underscores in the variable
names, such as \texttt{-14.29\_-4.76\_-0.79\_0.0\_50\_pts} or
\texttt{-14.29\_-4.76\_-0.79\_0.0\_50\_r} and the value in each cell
refers to the median r value or number of points, it was very difficult
to see what each column actually represented. The ideal format of the
data set was to have the sample index, minimum and maximum Alpha,
minimum and maximum Theta, number of points, median r, and type of
sample each be its own column. Hence, three key functions were used from
the tidyr package: gather, separate, and spread. The gather function
separated the dataset into key and value pairs for each index. The key
was the column name containing all essential information connected by
underscores and the value included the number of points or median r
value. Then, the separate function separated the result from the gather
function divided the column connected by underscore into 5 different
columns, named as \texttt{min\_alpha}, \texttt{max\_alpha},
\texttt{min\_theta}, \texttt{max\_theta}, and \texttt{ptsOrR}. This was
added to the result of the gather function that contained the index and
value of each cell, either median R or number of points. Finally, the
spread function widened the already wide table by expanding the
\texttt{ptsOrR} column by creating two columns, each column representing
median R and the number of points.

\subsection{Missing Value}\label{missing-value}

Samples with missing values are eliminated by Supporting vector machine.
For wedges that do not have any point in it, \texttt{median\ r} cannot
be calculated, which means that these sample will be eliminated when
running SVM. Wedges without points have biologically meanings, and we
should not ignore these wedges in our model. In order to keep the wedges
in our model, we need to artificially pick a \texttt{median\ r} value to
replace the missing ones. Supporting vector machine is sensitive to
outliers, so we cannot pick an \texttt{r} value that could become
outliers. We decided to calculate the mean of \texttt{median\ r} for the
nth landmark of all 52 samples, and then we replace the missing
\texttt{median\ r} values with the mean.

\section{Supporting Vector Machine}\label{supporting-vector-machine}

\subsection{What is SVM?}\label{what-is-svm}

(SH) SVM's have been proven to be a powerful algorithm for supervised
clustering. A good separation is achieved by the hyperplane that has the
largest distance to the nearest training-data point of any class.

\subsection{Implementation}\label{implementation}

\subsubsection{Cross-Validation}\label{cross-validation}

(DT) \#\# Result () \#\# Visulization ()

\section{Conclusion}\label{conclusion}

() \#\# Discussion () Strength Limitation

\subsection{Future Study}\label{future-study}

() Improvement Other models

\section*{References}\label{references}
\addcontentsline{toc}{section}{References}

\section{Appendix}\label{appendix}

\subsection{Figures}\label{figures}

\subsection{Python code}\label{python-code}

\nolinenumbers


\end{document}

